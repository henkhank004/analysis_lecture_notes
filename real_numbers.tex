\section{Real Numbers}


\subsubsection*{Upper bound}
\udef Let $R$ be a totally ordered field. Let $A \subset R$. Let $M \in R$.
Then $M$ is an upperbound of $A$ if
\begin{equation*}
    \forall a \in A \left[ a \leq M \right].
\end{equation*}


\subsubsection*{Bounded from above}
\udef Let $R$ be a totally ordered field. Let $A \subset R$. We say $A$ is
bounded from above if $\exists M \in R \forall a \in A \left[ a \leq M \right]$.


\subsubsection*{Lower bound}
\udef Let $R$ be a totally ordered field. Let $A \subset R$. Let $M \in R$.
Then $M$ is a lowerbound of $A$ if
\begin{equation*}
    \forall a \in A \left[ a \geq M \right].
\end{equation*}


\subsubsection*{Bounded from below}
\udef Let $R$ be a totally ordered field. Let $A \subset R$. We say $A$ is
bounded from below if $\exists M \in R \forall a \in A \left[ a \geq M \right]$.


\subsubsection*{Completeness Axiom}
A totally ordered field $R$ is \ul{complete} if every
$A \subset R, A \neq \emptyset$ that is bounded from above has a least upper bound.


\subsubsection*{Supremum}
Let $A \subset \R$. Let $M \in \R$. We call $M$ the least upper bound of $A$ if both
\begin{enumerate}[(i)]
    \item $\forall a \in A \left[ a \leq M \right]$,
    \item $\forall L \in \R \forall a \in A \left[ a \leq L \implies M \leq L \right]$.
\end{enumerate}
The least upper bound is unique.

The function that gives the least upper bound of a set $A \subset \R$, which is
non-empty and bounded from above, is called the \ul{supremum} and is denoted as
\begin{equation*}
    \sup A.
\end{equation*}
Note that often the least upper bound itself is called the supremum.


\subsubsection*{Infemum}
Let $A \subset \R$. Let $m \in \R$. We call $m$ the greatest lowest bound of $A$ if both
\begin{enumerate}[(i)]
    \item $\forall a \in A \left[ a \geq m \right]$,
    \item $\forall l \in \R \forall a \in A \left[ a \leq l \implies m \geq l \right]$.
\end{enumerate}
The greatest lower bound is unique.

The function that gives the greatest lower bound of a set $A \subset \R$, which is
non-empty and bounded from below, is called the \ul{infemum} and is denoted as
\begin{equation*}
    \inf A.
\end{equation*}
Note that often the greatest lower bound itself is called the infemum.


\subsubsection*{Alt. Char. Supremum}
\uthm $Let A \subset \R$, suppose $A \neq \emptyset$ and bounded from above.
Let $M \in \R$. $\sup A = M$ iff,
\begin{enumerate}[(i)]
    \item $\forall a \in A \left[ a \leq M \right]$ ($M$ is an upper bound of $A$),
    \item $\forall \varepsilon > 0 \exists a \in A \left[ a > M - \varepsilon \right]$
\end{enumerate}


\subsubsection*{Alt. Char Infemum}
\uthm Let $A \subset \R$, suppose $A \neq \emptyset$ and is bounded from below.
Let $m \in \R$. $\inf A = m$ iff,
\begin{enumerate}[(i)]
    \item $\forall a \in A \left[ a \geq m \right]$ ($m$ is a lower bound of $A$),
    \item $\forall \varepsilon > 0 \exists a \in A \left[ a < m + \varepsilon \right]$.
\end{enumerate}


\subsubsection*{Density of the Real Numbers}
\uthm Let $x,y \in \R$ assume $x < y$. Then there exists $a \in \R$ s.t.
$x < a < y$, i.e.
\begin{equation*}
    \forall x, y \in \R \exists a \in \R \left[ x < y \implies x < a < y \right].
\end{equation*}


\subsubsection*{Archemedian Property 1}
\uthm Let $x \in \R$. Then there exists $n \in \N$ s.t. $x \leq n$, i.e.
\begin{equation*}
    \forall x \in \R \exists n \in \N \left[ x \leq n \right].
\end{equation*}


\subsubsection*{Archemedian Property 2}
\uthm Let $x,y \in \R$ assume $x < y$. Then there exists $q \in \Q$ s.t. $x < q < y$,
i.e.
\begin{equation*}
    \forall x, y \in \R \exists q \in \Q \left[ x < y \implies x < q < y \right].
\end{equation*}
This property is also known as the density of the rationals.


\subsubsection*{Archemedian Property 3}
\uthm Let $x,y \in \R$ assume $x < y$. Then there exists $r \in \R \setminus \Q$
s.t. $x < r < y$, i.e.
\begin{equation*}
    \forall x, y \in \R \exists r \in \R \setminus \Q \left[ x < y \implies x < r < y \right].
\end{equation*}
This property is also known as the density of the irrationals.


\subsubsection*{Computation Rules for suprema and infema}
Let $A, B \subset \R$. Let $\lambda \in \R$. Define,
\begin{align*}
    A + B &:= \left\{ a + b\ \middle|\ a \in A, b \in B \right\}\text{, and}\\
    \lambda A &:= \left\{ \lambda a\ \middle|\ a \in A \right\}.
\end{align*}
\uprop Let $A,B,C,D \subset \R$ by nonempty. Assume that $A,B$ are bounded from
above and $C,D$ are bounded from below. Let $\lambda \in \R, \lambda \geq 0$ Then
\begin{enumerate}[(i)]
    \item $\sup(A + B) = \sup A + \sup B$,
    \item $\inf(C + D) = \inf C + \inf D$,
    \item $\sup(\lambda A) = \lambda \sup A$,
    \item $\inf(\lambda C) = \lambda \inf C$,
    \item $\sup(-C) = - \inf C$,
    \item $\inf(-A) = - \sup A$.
\end{enumerate}


\subsubsection*{Newton's Binomial Theorom}
\uthm Let $a, b \in \R$, then
\[
    (a+b)^n = \sum_{k = 0}^{n} \binom{n}{k} a^k b^{n-k}
\]
With $\binom{n}{k}$ ($n$ choose $k$) defined as
\[
    \frac{n!}{k!(n-k)!}
\]