\section{Subsequences, lim sup, and lim inf}


\subsection{Index sequences and subsequences}
\subsubsection*{Index sequence}
\udef We say a sequence $n : \N \to \N$ is an \emph{index sequence}
if $n$ is strinctly increasing, i.e.
\[
    \forall k \in \N \left[ n_{k+1} > n_k \right].
\]

\subsubsection*{Subsequence}
\udef Let $X$. Let $a,b : \N \to X$. We call $b$ a \emph{subsequence} of $a$
if there exists an index sequence $n : \N \to \N$ such that $b = a \circ n$.

\subsection{Accumulation points and subsequences of converging sequences}
\subsubsection*{(Sequential) accumulation points}
\udef Let $(X, \dist)$ be a metric space. A point $p \in X$ is called
a \emph{accumulation point} of a sequence $a : \N \to X$ if there exists
an index sequence $n : \N \to \N$ such that the subsequence $a \circ n$
converges to $p$. 

\section{Subsequences of a converging sequence}
\uprop Let $(X, \dist)$ be a metric space. Let $a : \N \to X$ be convergent to
$p \in X$. Then every subsequence of $a$ is convergent to $p$.


\subsection{lim sup}
\ulem Let $a : \N \to \R$ be bounded from above. Assume $a$ does not diverge
to $-\infty$. Then the sequence $k \mapsto \sup_{n \geq k} a_n$ is bounded
from below.

\subsubsection*{Limit superior}
\udef Let $a : \N \to \R$. The sequence $k \mapsto \sup_{n \geq k} a_n$ is bounded
from below and has a limit, this limit is equal to the infemum of the sequence.
This limit is called the $\limsup$. That is,
\begin{align*}
    \limsup_{n \to \infty} a_n :&= \inf_{k \in \N} \sup_{n \geq k} a_n\\ 
        &= \lim_{k \to \infty} \left( \sup_{n \geq k} a_n \right).
\end{align*}

We can thus write the following equality,
\[
    \limsup_{l \to \infty} a_l = \begin{cases}
        \infty & \text{if  $a$ is not bounded from above},\\
        -\infty & \text{if $a$ diverges to $-\infty$},\\
        \lim_{l \to \infty} \sup \left\{ a_k \middle| k \geq l \right\}
            & \text{in other cases}.
    \end{cases}
\]

\newpage

\subsubsection*{Alt char. lim sup}
Let $a : \N \to \R$. Let $M \in \R$.\\
Then $\limsup_{k \to \infty} a_k = M$ if, and only if,
\begin{enumerate}[(i)]
    \item $\forall \varepsilon > 0\ \exists N \in \N\ \forall n \geq N \left[ a_n < M + \varepsilon \right]$,
    \item $\forall \varepsilon > 0\ \forall K \in \N\ \exists l \geq K \left[ a_l > M - \varepsilon \right]$.
\end{enumerate}

\subsubsection*{}
\uthm Let $a : \N \to \R$ be bounded from above. Assume $a$ does not diverge to
$-\infty$.\\ Then $\limsup_{l \to \infty} a_l$ is the \emph{maximum} of the set
of sequential accumulation points.

\subsubsection*{Bolzano-Weierstrass}
\uthm Every bounded, real-valued sequence has a subsequence that converges
in $(\R, \dist_\R)$.


\subsection{Limit inferior}
\ulem Let $a : \N \to \R$ be bounded from below. Assume $a$ does not diverge
to $\infty$. Then the sequence $k \mapsto \inf_{n \geq k} a_n$ is bounded
from above.

\subsubsection*{Limit inferior}
\udef Let $a : \N \to \R$. The sequence $k \mapsto \inf_{n \geq k} a_n$ is bounded
from above and has a limit, this limit is equal to the supremum of the sequence.
This limit is called the $\liminf$. That is,
\begin{align*}
    \liminf_{n \to \infty} a_n :&= \sup_{k \in \N} \inf_{n \geq k} a_n\\ 
        &= \lim_{k \to \infty} \left( \inf_{n \geq k} a_n \right).
\end{align*}

We can thus write the following equality,
\[
    \liminf_{l \to \infty} a_l = \begin{cases}
        -\infty & \text{if  $a$ is not bounded from below},\\
        \infty & \text{if $a$ diverges to $\infty$},\\
        \lim_{l \to \infty} \inf \left\{ a_k \middle| k \geq l \right\}
            & \text{in other cases}.
    \end{cases}
\]

\subsubsection*{Alt char. lim inf}
Let $a : \N \to \R$. Let $M \in \R$. Then $M = \liminf_{l \to \infty} a_l$ if,
and only if,
\begin{enumerate}[(i)]
    \item $\forall \varepsilon > 0\ \exists N \in \N\ \forall n \geq N \left[ a_n > M - \varepsilon \right]$,
    \item $\forall \varepsilon > 0\ \forall K \in \N\ \exists l \geq K \left[ a_n < M + \varepsilon \right]$.
\end{enumerate}

\subsubsection*{}
\uthm Let $a : \N \to \R$ be bounded from below. Assume $a$ does not diverge to
$\infty$.\\ Then $\liminf_{l \to \infty} a_l$ is the \emph{minimum} of the set
of sequential accumulation points.

\newpage

\subsection{Relations between lim, lim sup, and lim inf}
\uprop Let $a : \N \to \R$. Let $L \in R$. Then $a$ converges to $L$ if,
and only if
\[
    \liminf_{l \to \infty} a_l = \limsup_{l \to \infty} a_l = L.
\]
Moreover if the above condition is true, then the $\lim_{l \to \infty} a_l$
exists and
\[
    \lim_{l \to \infty} a_l = \liminf_{l \to \infty} a_l = \limsup_{l \to \infty} a_l = L.
\]

\subsubsection*{}
\uprop Let $a,b : \N \to \R$. Assume that $\exists N \in \N\ \forall n \geq N
\left[ a_n \leq b_n \right]$. Then
\begin{align*}
    \limsup_{l \to \infty} a_l \leq &\limsup_{l \to \infty} b_l, \text{ and}\\
    \liminf_{l \to \infty} a_l \leq &\liminf_{l \to \infty} b_l.
\end{align*}