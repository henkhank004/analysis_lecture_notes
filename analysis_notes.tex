\documentclass[fleqn]{article}

\usepackage[utf8]{inputenc}
\usepackage{enumerate}
\usepackage{mathtools}
\usepackage{amsmath}
\usepackage{amsfonts}
\usepackage{amsthm}
\usepackage{amssymb}
\usepackage{stmaryrd}
\usepackage{mathrsfs}
\usepackage{titlesec}
\usepackage{hyperref}
\usepackage{marvosym}
\usepackage{wasysym}
\usepackage{xcolor}

\usepackage[a4paper, margin=3cm]{geometry}

\newcommand{\N}{\mathbb{N}}
\newcommand{\Z}{\mathbb{Z}}
\newcommand{\Q}{\mathbb{Q}}
\newcommand{\R}{\mathbb{R}}

\newcommand{\ul}[1]{\underline{#1}}

\newcommand{\udef}{\textbf{Definition:} }
\newcommand{\uthm}{\textbf{Theorom:} }
\newcommand{\uprop}{\textbf{Proposition:} }
\newcommand{\ucol}{\textbf{Corollary:} }
\newcommand{\ulem}{\textbf{Lemma:} }
\newcommand{\warning}{\textbf{\textcolor{red}{Warning: }}}

\newcommand{\dist}{\mathsf{dist}}
\newcommand{\norm}{\| \hspace*{2px} . \hspace*{2px} \|}
\newcommand{\distnorm}{\dist_{\norm}}
\newcommand{\normR}{| \hspace*{2px} . \hspace*{2px} |}

\newcommand{\setint}{\mathsf{int \hspace*{2px}}}

\newcommand{\blin}{\mathsf{BLin}}
\newcommand{\opnorm}[1]{\| #1 \|_{V \to W}}
\newcommand{\opnormd}{\norm_{V \to W}}


\titlespacing*{\subsection}
{5pt} % left margin
{4em} % space above
{1em} % space below

\title{2MBA40 \& 2MBA50: Analysis Not,es}
\author{}
\date{January 2025}

\begin{document}

\maketitle

The full lecture notes were made and are curated by dr. J.W. Portegies and may be found
\href{https://gitlab.tue.nl/jim-portegies/analysis}{here}. This merely
serves as a summary of the definitions, theoroms, and propositions
without proofs (mostly). Note that chapter numbers do \ul{not} match between
the original and this summary and that this documant may \ul{not} contain
all definitions, theoroms, and propositions.

Examples and proofs can be found in the original lecture notes for 2MBA40 and 2MBA50.


\section{Sets, Spaces, and Functions}

\subsection{Distance}
\udef Let $X$ be a set. A function $d \colon X \times X \to \mathbb{R}$
is a \ul{distance} if it satisfies all of the following properties:
\begin{enumerate}[(i)]
    \item (positivity)\\ $\forall x,y \in X \left[ d(x,y) \geq 0 \right]$,
    \item (symmmetry)\\ $\forall x,y \in X \left[ d(x,y) = d(y,x) \right]$,
    \item (triangle inequality)\\ $\forall x,y,z \in X \left[
        d(x,z) \leq d(x,y) + d(y,z) \right]$,
    \item (non-degeneracy)\\ $\forall x,y \in X \left[ d(x,y) = 0 \iff x = y \right]$,
    \item (reflexicity, result of (iv))\\ $\forall x \in X \left[ d(x,x) = 0 \right]$.
\end{enumerate}


\subsection{Metric Space}
\udef Let $X$ be a set. Let $\text{dist}_X \colon X \times X \to \mathbb{R}$
be a distance.

We call $(X, \text{dist}_X)$ a \ul{metric space}.


\subsection{Vector Space}
\udef A \ul{vector space} over a field $\mathbb{K}$
(note that usually, but not exclusively, $\mathbb{K} = \mathbb{R}$ or $\mathbb{K} = \mathbb{C}$)
is a set $V$ together with an element $0 \in V$ and operators $+\ \colon V \times V \to V$
(addition) and $\cdot\ \colon \mathbb{K} \times V \to V$ (scalar multiplication)
With the properties,
\begin{enumerate}[(i)]
    \item (commutativity of addition)\\
        $\forall u,v \in V \left[ u v = v + u \right]$,
    \item (associativity of addition)\\
        $\forall u,v,w \in V \left[ u+(v+w) = (u+v)+w \right]$,
    \item (0 as identity for addition)\\
        $\forall v \in V \left[ v + 0 = v \right]$,
    \item (1 as the identity of scalar multiplication)\\
        $\forall v \in V \left[ 1 \cdot v = v \right]$,
    \item (associativity of scalar multiplication)\\
        $\forall \lambda,\mu \in \mathbb{K} \forall v \in V \left[
            \lambda \cdot (\mu \cdot v) = (\lambda \cdot \mu) \cdot v \right]$,
    \item (distributivity of scalar multiplication over addition)\\
        $\forall u,v \in V \forall \lambda \in \mathbb{K} \left[
        \lambda \cdot (v + w) = \lambda \cdot v + \lambda \cdot w \right]$,
    \item (distributivity of scalar multiplication)\\
        $\forall \lambda, \mu \in \mathbb{K} \forall v \in V \left[
            (\lambda + \mu) \cdot \lambda = \lambda \cdot v + \mu \cdot v \right]$.
\end{enumerate}
May be denoted as $(V, +, \cdot, 0)$.
Note that in this course it will be assumed that any vector space is over the field
$\mathbb{R}$ unless stated otherwise.


\subsection{Norm}
\udef Let $V$ be a vector space. We say that the function
$\norm \colon V \to \mathbb{K}$ a \ul{norm} on $V$ if it satisfies
\begin{enumerate}[(i)]
    \item (positivity)\\ $\forall v \in V \left[ \| v \| \geq 0 \right]$,
    \item (non-degeneracy)\\ $\forall v \in V \left[ \| v \| = 0 \iff v = 0 \right]$,
    \item (absolute homogeneity)\\ $\forall v \in V \forall \lambda \in \mathbb{K}
        \left[ \| \lambda v \| = | \lambda | \| v \| \right]$,
    \item (triangle inequality)\\ $\forall u, v \in V \left[ 
        \| u + v \| \leq \| u \| + \| v \| \right]$.
\end{enumerate}


\subsection{Normed Vector Space}
\udef Let $V$ be a vector space.
Let $\norm_V\ \colon V \to \mathbb{K}$ be a norm on.
We call $(V, \norm_V)$ a \ul{normed vector space}.


\subsection{Matric Spaces from Normed Vector Spaces}
\uprop Any normed vector space gives rise to a metric space.
In particular. Let $(V, \norm)$ be a normed vector space. Define
the function $\distnorm \colon V \times V \to \mathbb{K}$ as
\[
    \distnorm(u,v) := \| u - v \|.
\]
Then $\distnorm$ is a distance on $V$. Therefore $(V, \distnorm)$ is a metric space.


\subsection{(Reverse) triangle inequality}
Let $(V, \norm)$ be a normed vector space. 
Then the following inequalities hold
\begin{align*}
    & \forall u,v \in V \left[ \| u + v \| \leq \| u \| + \| v \| \right],\\
    & \forall u,v \in V \left[ | \| u \| - \| v \| | \leq \| u - v \| \right].
\end{align*}

\section{Real Numbers}


\subsection{Upper bound}
\udef Let $R$ be a totally ordered field. Let $A \subset R$. Let $M \in R$.
Then $M$ is an upperbound of $A$ if
\begin{equation*}
    \forall a \in A \left[ a \leq M \right].
\end{equation*}


\subsection{Bounded from above}
\udef Let $R$ be a totally ordered field. Let $A \subset R$. We say $A$ is
bounded from above if $\exists M \in R \forall a \in A \left[ a \leq M \right]$.


\subsection{Lower bound}
\udef Let $R$ be a totally ordered field. Let $A \subset R$. Let $M \in R$.
Then $M$ is a lowerbound of $A$ if
\begin{equation*}
    \forall a \in A \left[ a \geq M \right].
\end{equation*}


\subsection{Bounded from below}
\udef Let $R$ be a totally ordered field. Let $A \subset R$. We say $A$ is
bounded from below if $\exists M \in R \forall a \in A \left[ a \geq M \right]$.


\subsection{Completeness Axiom}
A totally ordered field $R$ is \ul{complete} if every
$A \subset R, A \neq \emptyset$ that is bounded from above has a least upper bound.


\subsection{Supremum}
Let $A \subset \R$. Let $M \in \R$. We call $M$ the least upper bound of $A$ if both
\begin{enumerate}[(i)]
    \item $\forall a \in A \left[ a \leq M \right]$,
    \item $\forall L \in \R \forall a \in A \left[ a \leq L \implies M \leq L \right]$.
\end{enumerate}
The least upper bound is unique.

The function that gives the least upper bound of a set $A \subset \R$, which is
non-empty and bounded from above, is called the \ul{supremum} and is denoted as
\begin{equation*}
    \sup A.
\end{equation*}
Note that often the least upper bound itself is called the supremum.


\subsection{Infemum}
Let $A \subset \R$. Let $m \in \R$. We call $m$ the greatest lowest bound of $A$ if both
\begin{enumerate}[(i)]
    \item $\forall a \in A \left[ a \geq m \right]$,
    \item $\forall l \in \R \forall a \in A \left[ a \leq l \implies m \geq l \right]$.
\end{enumerate}
The greatest lower bound is unique.

The function that gives the greatest lower bound of a set $A \subset \R$, which is
non-empty and bounded from below, is called the \ul{infemum} and is denoted as
\begin{equation*}
    \inf A.
\end{equation*}
Note that often the greatest lower bound itself is called the infemum.


\subsection{Alt. Char. Supremum}
\uthm $Let A \subset \R$, suppose $A \neq \emptyset$ and bounded from above.
Let $M \in \R$. $\sup A = M$ iff,
\begin{enumerate}[(i)]
    \item $\forall a \in A \left[ a \leq M \right]$ ($M$ is an upper bound of $A$),
    \item $\forall \varepsilon > 0 \exists a \in A \left[ a > M - \varepsilon \right]$
\end{enumerate}


\subsection{Alt. Char Infemum}
\uthm Let $A \subset \R$, suppose $A \neq \emptyset$ and is bounded from below.
Let $m \in \R$. $\inf A = m$ iff,
\begin{enumerate}[(i)]
    \item $\forall a \in A \left[ a \geq m \right]$ ($m$ is a lower bound of $A$),
    \item $\forall \varepsilon > 0 \exists a \in A \left[ a < m + \varepsilon \right]$.
\end{enumerate}


\subsection{Density of the Real Numbers}
\uthm Let $x,y \in \R$ assume $x < y$. Then there exists $a \in \R$ s.t.
$x < a < y$, i.e.
\begin{equation*}
    \forall x, y \in \R \exists a \in \R \left[ x < y \implies x < a < y \right].
\end{equation*}


\subsection{Archemedian Property 1}
\uthm Let $x \in \R$. Then there exists $n \in \N$ s.t. $x \leq n$, i.e.
\begin{equation*}
    \forall x \in \R \exists n \in \N \left[ x \leq n \right].
\end{equation*}


\subsection{Archemedian Property 2}
\uthm Let $x,y \in \R$ assume $x < y$. Then there exists $q \in \Q$ s.t. $x < q < y$,
i.e.
\begin{equation*}
    \forall x, y \in \R \exists q \in \Q \left[ x < y \implies x < q < y \right].
\end{equation*}
This property is also known as the density of the rationals.


\subsection{Archemedian Property 3}
\uthm Let $x,y \in \R$ assume $x < y$. Then there exists $r \in \R \setminus \Q$
s.t. $x < r < y$, i.e.
\begin{equation*}
    \forall x, y \in \R \exists r \in \R \setminus \Q \left[ x < y \implies x < r < y \right].
\end{equation*}
This property is also known as the density of the irrationals.


\subsection{Computation Rules for suprema and infema}
Let $A, B \subset \R$. Let $\lambda \in \R$. Define,
\begin{align*}
    A + B &:= \left\{ a + b\ \middle|\ a \in A, b \in B \right\}\text{, and}\\
    \lambda A &:= \left\{ \lambda a\ \middle|\ a \in A \right\}.
\end{align*}
\uprop Let $A,B,C,D \subset \R$ by nonempty. Assume that $A,B$ are bounded from
above and $C,D$ are bounded from below. Let $\lambda \in \R, \lambda \geq 0$ Then
\begin{enumerate}[(i)]
    \item $\sup(A + B) = \sup A + \sup B$,
    \item $\inf(C + D) = \inf C + \inf D$,
    \item $\sup(\lambda A) = \lambda \sup A$,
    \item $\inf(\lambda C) = \lambda \inf C$,
    \item $\sup(-C) = - \inf C$,
    \item $\inf(-A) = - \sup A$.
\end{enumerate}


\subsection{Newton's Binomial Theorom}
\uthm Let $a, b \in \R$, then
\[
    (a+b)^n = \sum_{k = 0}^{n} \binom{n}{k} a^k b^{n-k}
\]
With $\binom{n}{k}$ ($n$ choose $k$) defined as
\[
    \binom{n}{k} := \frac{n!}{k!(n-k)!}
\]
\section{Sequences}


\subsection{Sequence}
\udef Let $X$ be a set. A \emph{sequence} $a : \N \to X$ is a function
from the natural numbers $\N$ to $X$.


\subsection{Bounded sequence}
\udef Let $(X, \dist)$ be a metric space.
We say a sequence $a : \N \to X$ is \emph{bounded} if
\[
    \exists q \in X\ \forall M \geq 0\ \forall n \in \N \left[ 
    \dist(a_n, q) \leq M \right].
\]


\subsection{Bounded sequence in a normed vector space}
\udef Let $(V, \norm)$ be a normed vector space.
Let $a : \N \to V$. The sequence $a$ is bounded if, and only if,
\[
    \exists M \geq 0\ \forall n \in \N \left[ \| a_n \| \leq M \right].
\]


\subsection{Convergence}
Let $(X, \dist)$ be a metric space. Let $q \in X$. Let $a : \N \to X$.
$a$ \emph{converges} to $p$ if
\[
    \forall \varepsilon > 0\ \exists N \in \N\ \forall n \geq N \left[
    \dist(a_n,p) < \varepsilon \right].
\]
We often write
\[
    \lim_{n \to \infty}a_n = p.
\]


\subsection{Uniqueness of limits}
Let $(X, \dist)$ be a metric space. Let $p,q \in X$. Let $a : \N \to X$.
Assume both $a$ converges to $p$ \emph{and} $a$ converges to $q$, then $p = q$ i.e.,
\[
    \left( \lim_{n \to \infty} a_n = p \land \lim_{n \to \infty} a_n = q \right)
    \implies p = q.
\]


\subsection{Limit Theoroms}
Let $(V, \norm)$ be a normed vector space.\\
Let $a,b : \N \to V$. Let $\lambda : \N \to \R$.\\
Let $p,q \in V$ Let $\mu \in \R$.\\
Assume that
\begin{align*}
    \lim_{n \to \infty} a_n &= p,\\
    \lim_{n \to \infty} b_n &= q,\\
    \lim_{n \to \infty} \lambda_n &= \mu.
\end{align*}
Then
\begin{enumerate}
    \item $\lim_{n \to \infty} (a_n + b_n) = p + q$,
    \item $\lim_{n \to \infty} (\lambda_n \cdot a_n) = \mu \cdot p$.
\end{enumerate}


\subsection{Index shift}
Ket $(X, \dist)$ be a metric space. Let $a : \N \to X$. Let $k \in \N$.\\
Define $b : \N \to X$ as $b_n := a_{n+k}$.\\
Then $a$ converges to $q$ if, and only if, $b$ converges to $a$
\section{Real valued sequences}



\subsection{Monotone Sequences}
\udef Let $a : \N \to \R$. We call $a$ \emph{non-decreasing} if
\[
    \forall n \in \N \left[ a_{n+1} \geq a_n \right]
\]
\udef We call $a$ \emph{strictly increasing} if
\[
    \forall n \in \N \left[ a_{n+1} > a_n \right]
\]
\udef Let $a : \N \to \R$. We call $a$ \emph{non-increasing} if
\[
    \forall n \in \N \left[ a_{n+1} \leq a_n \right]
\]
\udef We call $a$ \emph{strictly decreasing} if
\[
    \forall n \in \N \left[ a_{n+1} < a_n \right]
\]
\udef We call $a$ bounded from above if
\[
    \exists M \in \R\ \forall n \in \N \left[ a_n \leq M \right].
\]
\udef We call $a$ bounded from below if
\[
    \exists M \in \R\ \forall n \in \N \left[ a_n \geq M \right].
\]


\subsection{Every bounded, monotone real-valued sequence converges}
\udef For every $\R$-valued sequence, we call a sequence \emph{monotone} if
it is non-decreasing or non-increasing.

\subsubsection*{}
\uthm Let $a : \N \to \R$ be bounded from above and non-decreasing. Then $a$
\emph{converges} and
\[
    \lim_{n \to \infty} a_n = \sup_{n \in \N} a_n
        := \sup \left\{ a_n\ \middle|\ n \in \N \right\}.
\]

\subsubsection*{}
\uthm Let $a : \N \to \R$ be bounded from below and non-increasing. Then $a$
\emph{converges} and 
\[
    \lim_{n \to \infty} a_n = \inf_{n \in \N} a_n
        := \inf \left\{ a_n\ \middle|\ n \in \N \right\}.
\]


\subsection{Limit Theoroms}
\uthm Let $a,b : \N \to \R$. Let $c,d \in \R$.\\
Assume
\begin{align*}
    \lim_{n \to \infty} a_n &= c \text{, and}\\
    \lim_{n \to \infty} b_n &= d.
\end{align*}
Then
\begin{enumerate}
    \item $\lim_{n \to \infty} ( a_n + b_n ) = c + d$,
    \item $\lim_{n \to \infty} ( a_n \cdot b_n) = c \cdot d$,
    \item if $d \neq 0$, then $\lim_{n \to \infty} \frac{a_n}{b_n} = \frac{c}{d}$,
    \item $\forall m \in \N \left[ \lim_{n \to \infty} (a_n)^m = c^m \right]$,
    \item $\forall n \in \N\ \forall k \in \N \setminus \{0\} \left[ a_n \geq 0 \implies
        \lim_{n \to \infty} (a_n)^{\frac{1}{k}} = c^{\frac{1}{k}} \right]$
\end{enumerate}
Combining (iv) and (v) yields
\[
    \forall n \in \N\ \forall q \in \Q \left[
        a_n \geq 0 \implies \lim_{n \to \infty} (a_n)^q = c^q \right].
\]


\subsection{Squeeze Theorom}
\uthm Let $a,b,c : \N \to \R$.\\
Assume that $\exists N \in \N\ \forall n \geq N \left[ a_n \leq b_n \leq c_n \right]$.\\
Assume that $a$ and $c$ converge, moreover assume
    $\lim_{n \to \infty} a_n = \lim_{n \to \infty} c_n$.\\
Then also
\[
    \lim_{n \to \infty} a_n = \lim_{n \to \infty} b_n = \lim_{n \to \infty} c_n.
\]


\subsection{Divergence to $\infty$ and $-\infty$}
\udef Let $a : \N \to \R$ we say $a$ diverges to $\infty$ and write
$\lim_{n \to \infty} a_n = \infty$ if
\[
    \forall M \in \R\ \exists N \in \N\ \forall n \geq N \left[
    a_n > M \right].
\]
Similarly, we say $a$ diverges to $-\infty$ and write
$\lim_{n \to \infty} a_n = -\infty$ if
\[
    \forall M \in \R\ \exists N \in \N\ \forall n \geq N \left[
    a_n < M \right].
\]

\subsubsection*{}
\uprop Let $a : \N \to \R$ be a sequence diverging to $\infty$.\\
Then $a$ is bounded from below.\\

Let $b : \N \to \R$ be a sequence diverging to $-\infty$.\\
Then $b$ is bounded from above.\\


\subsection{Limit theoroms for improper limits}
\uthm Let $a,b,c,d : \N \to \R$. Assume
\begin{align*}
    &\lim_{n \to \infty} a_n = \infty,\\
    &\lim_{n \to \infty} c_n = -\infty,\\
    &\text{$b$ is bounded from below},\\
    &\text{$d$ is bounded from above}.
\end{align*}
Let $\lambda : \N \to \R$ be bounded from below by some $\mu > 0$.
Then
\begin{enumerate}
    \item $\lim_{n \to \infty} (a_n + b_n) = \infty$,
    \item $\lim_{n \to \infty} (c_n + d_n) = -\infty$,
    \item $\lim_{n \to \infty} (\lambda_n a_n) = \infty$,
    \item $\lim_{n \to \infty} (\lambda_n c_n) = -\infty$.
\end{enumerate}


\subsubsection*{}
\uprop Let $a : \N \to \R$. Let $b : \N \to (0, \infty)$ Then,
\begin{enumerate}
    \item $\lim_{n \to \infty} a_n = \infty \iff \lim_{n \to \infty} (-a_n) = -\infty$,
    \item $\lim_{n \to \infty} b_n = \infty \iff \lim_{n \to \infty} \frac{1}{b_n} = 0$.
\end{enumerate}


\subsection{Standard sequences}

\subsubsection*{Geometric Sequence}
\uprop Let $q \in \R$ Define $a : \N \to \R$ by $a_n := q^n$
\[
    \lim_{n \to \infty} a_n = \lim_{n \to \infty} q^n =
    \begin{cases}
        \infty & \text{if } q > 1,\\
        1 & \text{if } q = 1,\\
        0 & \text{if } q \in (-1, 1),\\
        \text{diverges} & \text{if } q \leq -1.
    \end{cases}
\]

\subsection*{The $n^{\text{th}}$ root of $n$}
\uprop The sequence $n \mapsto \sqrt[n]{n}$ converges to $1$.

\subsection*{Definition of $e$}
\udef We define
\[
    e := \lim_{n \to \infty} \left( 1 + \frac{1}{n} \right)^n
\]

\subsection*{Exponentials beat powers}
\uprop Let $a \in (1, \infty)$ and let $p \in (0, \infty)$. Then
\[
    \lim_{n \to \infty} \frac{n^p}{a^n} = 0.
\]


\subsection{$\R^d$-valued sequences}
\uprop Consider the metric space $(\R^d, \norm_2)$. Let $z \in \R^d$.\\
Let $x : \N \to \R^d$. We will also denote this sequence with $(x^{(n)})$.\\
Denote by $y_i$ the $i^{\text{th}}$ component of a vector $y \in \R^d$.\\
The sequence $(x^{(n)})$ converges to $z$ if, and only if,
\[
    \forall i \in {1, \dots, d} \left[ \lim_{n \to \infty} (x_i^{(n)}) = z_i \right].
\]
\section{Series}

\subsection{Definition}
\udef Let $(V, \norm)$ be a normed vector space.\\
Let $a : \N \to V$. Let $K \in \N$.\\
We say that a series
\[
    \sum_{n=K}^{\infty} a_n
\]
is \emph{convergent} if the associated \emph{sequence of partial sums}
$S_K : \N \to V$ converges.
The term $S_K^n$ is, for $n \in \N$, defined as
\[
    S_K^n := \sum_{k=K}^{n} a_k
\]
If $K = 0$, we usually write $S^n$ or $S_n$ instead of $S_0^n$.
If the series $\sum_{n=K}^{\infty}$ is convergent, the \emph{value} of the series
is defined as the limit of the sequence of partial sums, i.e.
\[
    \sum_{k=K}^{\infty} a_k := \lim_{n \to \infty} S_K^n
        = \lim_{n \to \infty} \sum_{k=K}^{n} a_k.
\]

\subsection{Geometric series}
\uprop Let $a \neq 1$ and $n \in \N$. Then
\[
    \sum_{k=0}^{n} a^k = \frac{1 - a^{n+1}}{1 - a}.
\]

\subsubsection*{}
\uprop Let $a \in (-1, 1)$. Then
\[
    \sum_{k=0}^{\infty} a^k = \frac{1}{1-a}.
\]

\subsection{Harmonic Series}
\uprop The series
\[
    \sum_{k=1}^{\infty} \frac{1}{k}
\]
diverges.

\subsection{Hyperharmonic Series}
\uprop Let $p > q$ then the series
\[
    \sum_{k=1}^{\infty} \frac{1}{k^p}
\]
converges.

\subsubsection*{Basel Problem}
\uthm
\[
    \sum_{n=1}^{\infty} \frac{1}{n^2} = \frac{\pi^2}{6}
\]


\subsection{Only the tail matters for convergence}
\uprop Let $(V, \norm)$ be a normed vector space.\\
Let $a : \N \to V$. Let $K,L \in \N$. Then the series
\[
    \sum_{n=K}^{\infty} a_n
\]
converges if, and only if,
\[
    \sum_{n=L}^{\infty} a_n
\]
converges. Moreover, if they converge and $K < L$, then
\[
    \sum_{n=K}^{\infty} a_n = \sum_{n=K}^{L-1} a_n + \sum_{n=K}^{\infty} a_n.
\]

\subsubsection*{}
\uprop Let $(V, \norm)$ be a normed vector space.\\
Let $a : \N \to V$. Let $M \in \N$. Assume the series
\[
    \sum_{k=M}^{\infty} a_k
\]
is convergent. Then
\[
    \lim_{m \to \infty} \sum_{k=m}^{\infty} a_k = 0.
\]

\subsubsection*{Index shift for sequences}
Let $(V, \norm)$ be a normed vector space.\\
Let $a : \N \to V$, let $M, l \in \N$. Then the series
\[
    \sum_{k=M}^{\infty} a_k
\]
converges if, and only if,
\[
    \sum_{k=M}^{\infty} a_{k+l}
\]
converges. Moreover, if they converge,
\[
    \sum_{k=M}^{\infty} a_{k+l} = \sum_{k=M+l}^{\infty} a_k.
\]


\subsection{Divergence test}
\uthm Let $(V, \norm)$ be a normed vector space.\\
Let $a : \N \to V$. Suppose the series $\sum_{n=0}^{\infty} a_n$ converges.
Then
\[
    \lim_{n \to \infty} a_n = 0.
\]
In other words: if the limit $\lim_{n \to \infty} a_n$ diverges or is not
equal to $0$, then the series $\sum_{n=0}^{\infty} a_n$ diverges.\\
\warning This condition is necessary but \emph{not} sufficient.


\subsection{Limit laws for series}
\uthm Let $(V, \norm)$ be a normed vector space.\\
Let $a,b : \N \to V$. Let $\lambda \in \R$.\\
Suppose $\sum_{k=0}^{\infty} a_k$ and $\sum_{k=0}^{\infty} b_k$ converge.\\
Then the following series converge and are equal to,
\begin{enumerate}[(i)]
    \item $\sum_{k=0}^{\infty} (a_k + b_k) = \sum_{k=0}^{\infty} a_k + \sum_{k=0}^{\infty} a_k$,
    \item $\sum_{k=0}^{\infty} (\lambda a_k) = \lambda \sum_{k=0}^{\infty} a_k$.
\end{enumerate}
\section{Series with positive terms}
\emph{Note} that in this chapter positive will often be used instead
of non-negative for simplicity.

\subsubsection*{}
Let $a_n : \N \to [0,\infty)$ be a positive real-values sequence.\\
Consider the series with positive real values, $\sum_{n=0}^{\infty} a_n$.\\
Then the associated sequence of partial sums $(S^{(n)})$ is non-decreasing,
\[
    S_{n+1} - S_n = a_{n+1} \geq 0.
\]
Therefore there are only two posibilities for the limit,
\begin{enumerate}
    \item $\lim_{n \to \infty} S_n$ exists, or
    \item $\lim_{n \to \infty} S_n = \infty$.
\end{enumerate}

\subsection{Comparison test}
\uthm Let $a,b : \N \to [0,\infty)$. Assume that
\[
    \exists N \in \N\ \forall n \geq N \left[ a_n \leq b_n \right].
\]
Then
\begin{enumerate}[(i)]
    \item Suppose $\sum_{n=0}^{\infty} b_k$ converges,\\
        then also $\sum_{n=0}^{\infty} a_k$ converges.
    \item Suppose $\sum_{n=0}^{\infty} a_k$ diverges,\\
        then also $\sum_{n=0}^{\infty} b_k$ diverges.
\end{enumerate}

\subsection{Limit comparison test}
\uthm Let $a,b : \N \to [0,\infty)$.
\begin{enumerate}
    \item Assume the series $\sum_{n=0}^{\infty} b_n$ converges, but not to $0$,
        and assume the limit \[
            \lim_{n \to \infty} \frac{a_n}{b_n} 
        \] exists (i.e. the sequence $n \mapsto a_n / b_n$ converges).\\
        Then the series $\sum_{n=0}^{\infty} a_n$ also converges.
    \item Assume the series $\sum_{n=0}^{\infty} b_n$ diverges, and assume that 
    either \[
        \lim_{n \to \infty} \frac{a_n}{b_n} > 0,
    \] or that \[
        \lim_{n \to \infty} \frac{a_n}{b_n} = \infty.
    \] Then the series $\sum_{n=0}^{\infty} a_n$ also diverges.
\end{enumerate}

\subsection{Ratio test}
\uthm Let $a : \N \to (0, \infty)$.
\begin{enumerate}
    \item Assume that \[\exists q \in (0,1)\ \exists N \in \N\ \forall n \geq N
        \left[ \frac{a_{n+1}}{a_n} < q \right]\] then the series
        $\sum_{n=0}^{\infty} a_n$ converges.
    \item Assume \[ \exists N \in \N\ \forall n \geq N \left[ 
        \frac{a_{n+1}}{a_n} \geq 1 \right] \] then the series
        $\sum_{n=0}^{\infty} a_n$ diverges.
\end{enumerate}

\subsubsection*{Ratio test, limit version}
\ucol Let $a : \N \to (0, \infty)$.
\begin{itemize}
    \item if \[
            \lim_{n \to \infty} \frac{a_{n+1}}{a_n} \in [0,1)
        \] then the series $\sum_{k=0}^{\infty} a_k$ converges.
    \item if \[
            \lim_{n \to \infty} \frac{a_{n+1}}{a_n} \in (1,\infty) \text{ or }
            \lim_{n \to \infty} \frac{a_{n+1}}{a_n} = \infty
        \] then the series $\sum_{k=0}^{\infty} a_k$ diverges.
\end{itemize}
\warning We cannot conclude anything about the convergence of a series
$\sum_{k=0}^{\infty} a_k$ when \[
            \lim_{n \to \infty} \frac{a_{n+1}}{a_n} = 1.
\]

\subsection{Root test}
\uthm Let $a : \N \to [0,\infty)$.
\begin{enumerate}[(i)]
    \item Suppose \[
            \exists q \in (0,1)\ \exists N \in \N\ \forall n \geq N \left[ 
            \sqrt[n]{a_n} < q \right]
        \] then $\sum_{k=0}^{\infty} a_k$ converges.
    \item Suppose \[
            \exists N \in \N\ \forall n \geq N \left[ 
            \sqrt[n]{a_n} \geq 1 \right]
        \] then $\sum_{k=0}^{\infty} a_k$ diverges.
\end{enumerate}

\subsubsection*{Root test, limit version}
\ucol Let $a : \N \to [0, \infty)$.
\begin{enumerate}[(i)]
    \item if \[
            \lim_{n \to \infty} \sqrt[n]{a_n} \in [0,1)
        \] then the series $\sum_{k=0}^{\infty} a_k$ converges.
    \item if \[
            \lim_{n \to \infty} \sqrt[n]{a_n} \in (1,\infty) \text{ or }
            \lim_{n \to \infty} \sqrt[n]{a_n} = \infty
        \] then the series $\sum_{k=0}^{\infty} a_k$ diverges.
\end{enumerate}
\warning We cannot conclude anything about the convergence of a series
$\sum_{k=0}^{\infty} a_k$ when \[
            \lim_{n \to \infty} \sqrt[n]{a_n} = 1.
\]
\section{Series with general terms}


\subsection{Leibniz test}
\uthm Let $a,b : \N \to \R$ such that $b_n := (-1)^n a_n$.
Assume that there exists a $K \in \N$ such that
\begin{enumerate}[(i)]
    \item $\forall k \geq K \left[ a_k \geq 0 \right]$,
    \item $\forall k \geq K \left[ a_{k+1} \leq a_k \right]$,
    \item $\lim_{n \to \infty} a_n = 0$.
\end{enumerate}
Then the series 
\[
    \sum_{n=0}^{\infty} b_n = \sum_{n=0}^{\infty} (-1)^n a_n
\]
converges.

\subsubsection*{}
\uprop If conditions (i), (ii), (iii) in the Leibniz test hold.
Then for every $N \geq K$ and $S_n := \sum_{k=K}^{N} b_n$ the associated
sequence of partial sums,
\[
    \left| S_N - \sum_{k=K}^{\infty} b_k \right| < a_{N + 1}.
\]


\subsection{Series characterisation of completeness in a normed vector space}
\udef Let $(V, \norm)$ be a normed vector space. Let $a : \N \to V$.
We say the series
\[
    \sum_{k=0}^{\infty} a_k
\]
\emph{converges absolutely} if
\[
    \sum_{k=0}^{\infty} \| a_k \|
\]
converges.

\subsubsection*{Series characterisation of completeness}
\udef We say a normed vector space $(V, \norm)$ satisfies the
\emph{series characterisation of completeness} if every series in $V$
that is absolutely convergent is also convergent.

\udef Let $(V, \norm)$ be a normed vector space. Let $a : \N \to V$.
We say that the series
\[
    \sum_{k=0}^{\infty} a_k
\]
\emph{converges conditionally} if it converges, but does not converge absolutely.


\subsection{Cauchy product}
Let $a, b, c : \N \to \R$.\\
Define $c_k := \sum_{n}^{k=0} a_l b_{k-l}$.
Assume that the series
\[
    \sum_{k=0}^{\infty} a_k \text{ and }
    \sum_{k=0}^{\infty} b_k
\]
both converge absolutely. Then the following series also converges absolutely
\[
    \sum_{k=0}^{\infty} c_k = \left( \sum_{k=0}^{\infty} a_k \right)
        + \left( \sum_{k=0}^{\infty} b_k \right)
\]
\section{Subsequences, lim sup, and lim inf}


\subsection{Index sequences and subsequences}
\subsubsection*{Index sequence}
\udef We say a sequence $n : \N \to \N$ is an \emph{index sequence}
if $n$ is strinctly increasing, i.e.
\[
    \forall k \in \N \left[ n_{k+1} > n_k \right].
\]

\subsubsection*{Subsequence}
\udef Let $X$. Let $a,b : \N \to X$. We call $b$ a \emph{subsequence} of $a$
if there exists an index sequence $n : \N \to \N$ such that $b = a \circ n$.

\subsection{Accumulation points and subsequences of converging sequences}
\subsection{(Sequential) accumulation points}
\udef Let $(X, \dist)$ be a metric space. A point $p \in X$ is called
a \emph{accumulation point} of a sequence $a : \N \to X$ if there exists
an index sequence $n : \N \to \N$ such that the subsequence $a \circ n$
converges to $p$. 

\section{Subsequences of a converging sequence}
\uprop Let $(X, \dist)$ be a metric space. Let $a : \N \to X$ be convergent to
$p \in X$. Then every subsequence of $a$ is convergent to $p$.


\subsection{lim sup}
\ulem Let $a : \N \to \R$ be bounded from above. Assume $a$ does not diverge
to $-\infty$. Then the sequence $k \mapsto \sup_{n \geq k} a_n$ is bounded
from below.

\subsubsection*{Limit superior}
\udef Let $a : \N \to \R$. The sequence $k \mapsto \sup_{n \geq k} a_n$ is bounded
from below and has a limit, this limit is equal to the infemum of the sequence.
This limit is called the $\limsup$. That is,
\begin{align*}
    \limsup_{n \to \infty} a_n :&= \inf_{k \in \N} \sup_{n \geq k} a_n\\ 
        &= \lim_{k \to \infty} \left( \sup_{n \geq k} a_n \right).
\end{align*}

We can thus write the following equality,
\[
    \limsup_{l \to \infty} a_l = \begin{cases}
        \infty & \text{if  $a$ is not bounded from above},\\
        -\infty & \text{if $a$ diverges to $-\infty$},\\
        \lim_{l \to \infty} \sup \left\{ a_k \middle| k \geq l \right\}
            & \text{in other cases}.
    \end{cases}
\]

\newpage

\subsubsection*{Alt char. lim sup}
Let $a : \N \to \R$. Let $M \in \R$.\\
Then $\limsup_{k \to \infty} a_k = M$ if, and only if,
\begin{enumerate}[(i)]
    \item $\forall \varepsilon > 0\ \exists N \in \N\ \forall n \geq N \left[ a_n < M + \varepsilon \right]$,
    \item $\forall \varepsilon > 0\ \forall K \in \N\ \exists l \geq K \left[ a_l > M - \varepsilon \right]$.
\end{enumerate}

\subsubsection*{}
\uthm Let $a : \N \to \R$ be bounded from above. Assume $a$ does not diverge to
$-\infty$.\\ Then $\limsup_{l \to \infty} a_l$ is the \emph{maximum} of the set
of sequential accumulation points.

\subsubsection*{Bolzano-Weierstrass}
\uthm Every bounded, real-valued sequence has a subsequence that converges
in $(\R, \dist_\R)$.


\subsection{Limit inferior}
\ulem Let $a : \N \to \R$ be bounded from below. Assume $a$ does not diverge
to $\infty$. Then the sequence $k \mapsto \inf_{n \geq k} a_n$ is bounded
from above.

\subsubsection*{Limit inferior}
\udef Let $a : \N \to \R$. The sequence $k \mapsto \inf_{n \geq k} a_n$ is bounded
from above and has a limit, this limit is equal to the supremum of the sequence.
This limit is called the $\liminf$. That is,
\begin{align*}
    \liminf_{n \to \infty} a_n :&= \sup_{k \in \N} \inf_{n \geq k} a_n\\ 
        &= \lim_{k \to \infty} \left( \inf_{n \geq k} a_n \right).
\end{align*}

We can thus write the following equality,
\[
    \liminf_{l \to \infty} a_l = \begin{cases}
        -\infty & \text{if  $a$ is not bounded from below},\\
        \infty & \text{if $a$ diverges to $\infty$},\\
        \lim_{l \to \infty} \inf \left\{ a_k \middle| k \geq l \right\}
            & \text{in other cases}.
    \end{cases}
\]

\subsubsection*{Alt char. lim inf}
Let $a : \N \to \R$. Let $M \in \R$. Then $M = \liminf_{l \to \infty} a_l$ if,
and only if,
\begin{enumerate}[(i)]
    \item $\forall \varepsilon > 0\ \exists N \in \N\ \forall n \geq N \left[ a_n > M - \varepsilon \right]$,
    \item $\forall \varepsilon > 0\ \forall K \in \N\ \exists l \geq K \left[ a_n < M + \varepsilon \right]$.
\end{enumerate}

\subsubsection*{}
\uthm Let $a : \N \to \R$ be bounded from below. Assume $a$ does not diverge to
$\infty$.\\ Then $\liminf_{l \to \infty} a_l$ is the \emph{minimum} of the set
of sequential accumulation points.

\newpage

\subsection{Relations between lim, lim sup, and lim inf}
\uprop Let $a : \N \to \R$. Let $L \in R$. Then $a$ converges to $L$ if,
and only if
\[
    \liminf_{l \to \infty} a_l = \limsup_{l \to \infty} a_l = L.
\]
Moreover if the above condition is true, then the $\lim_{l \to \infty} a_l$
exists and
\[
    \lim_{l \to \infty} a_l = \liminf_{l \to \infty} a_l = \limsup_{l \to \infty} a_l = L.
\]

\subsubsection*{}
\uprop Let $a,b : \N \to \R$. Assume that $\exists N \in \N\ \forall n \geq N
\left[ a_n \leq b_n \right]$. Then
\begin{align*}
    \limsup_{l \to \infty} a_l \leq &\limsup_{l \to \infty} b_l, \text{ and}\\
    \liminf_{l \to \infty} a_l \leq &\liminf_{l \to \infty} b_l.
\end{align*}
\section{Point-set topology of metric spaces}

\subsection{Open sets}
\subsubsection*{Open ball}
\udef Let $(X, \dist)$ be a metric space. We define the open ball $B$ around
a point $p \in X$ with radius $r > 0$ by
\[
    B(p,r) := \left\{ x \in X \middle| \dist(x,p) < r \right\}.
\]

\subsubsection*{Interior point}
Let $(X, \dist)$ be a metric space.
Let $O \subset X$. Let $a \in O$.\\
Then $a$ is an \emph{interior point} of $O$ if
\[
    \exists r > 0 \left[ B(x,r) \subset O \right].
\]

\subsubsection*{Open subset}
\udef Let $(X, \dist)$ be a metric space.
Let $O \subset X$. Then $O$ is \emph{open} if
\[
    \forall x \in O\ \exists r > 0 \left[ B(x,r) \subset O \right].
\]

\subsubsection*{}
\uprop Let $(X, \dist)$ be a metric space. Then $X$ is open.

\subsubsection*{}
\uprop Let $(X, \dist)$ be a metric space. Then $\varnothing \subset X$ is open.

\subsubsection*{}
\uprop Let $(X, \dist)$ be a metric space. The open ball
\[
    B(p,q) := \left\{ x \in X \middle| \dist(x,p) < r \right\}.
\]
is indeed open.

\subsubsection*{'Open intervals' are open sets}
\uprop Let $a, b \in \R$, with $a < b$. Then the intervals $(a,b)$, $(a, \infty)$, and
$(-\infty, b)$ are all open subsets of $\R$ in the normed vector space
$(\R, \normR)$.

\subsubsection*{Interior of a set}
Let $(X, \dist)$ be a metric space. Let $A \subset X$. Then the \emph{interior}
of $A$, denoted by $\setint A$ is the set of all interior points of $A$, i.e.
\[
    \setint A := \left\{ x \in A\ \middle|\ x \text{ is an interior point of } A \right\}.
\]

\subsubsection*{}
\uprop Let $(X, \dist)$ be a metric space. Let $A \subset X$. Then $\setint A$
is open.

\subsubsection*{Union of open sets is always open}
\uthm Let $(X, \dist)$ be a metric space. Let $\mathcal{I}$.
Assume that for every $\alpha \in \mathcal{I}$, we have $O_\alpha \subset X$.
Suppose moreover that for all $\alpha \in \mathcal{I}$, $O_\alpha$ is open.
Then also
\[
    \bigcup_{\alpha \in \mathcal{I}}\ O_\alpha
\]
is open.

\subsubsection*{Finite intersections of open sets are open}
\uprop Let $(X, \dist)$ be a metric space. Let $O_1, \dots, O_N$ be open subsets
of $X$. Then the intersection
\[
    O_1 \cap \dots \cap O_N.
\]
is also open.

\subsubsection*{Cartesian product of open sets}
Let $O_1, \dots, O_d$ be open subsets of $\R$. Then
\[
    O_1 \times \dots \times O_d
\]
is an open subset of $(\R^d, \norm_2)$.

\subsection{Closed sets}
\udef Let $(X, \dist)$ be a metric space. We say a set $C \subset X$ is closed
if $X \setminus C$ is open.

\subsubsection*{Sequence characterization of closedness}
\uprop Let $(X, \dist)$ be a metric space. A set $C \subset X$ is closed if,
and only if,
\[
    \forall c : \N \to C\ \exists q \in X \left[ \lim_{n \to \infty} c_n = q \implies q \in C \right].
\]

\subsubsection*{}
\uprop Let $(X, \dist)$ be a metric space. Then $X$ is open.

\subsubsection*{}
\uprop Let $(X, \dist)$ be a metric space. Then $\varnothing \subset X$ is closed.

\subsubsection*{'Closed intervals' are closed sets}
\uprop Let $a, b \in \R$, with $a < b$. Then the interval $[a,b]$ is a closed
subsets of $\R$ in the normed vector space $(\R, \normR)$.\\
\emph{Note} that closed intervals are even compact.

\subsubsection*{Intersections of closed sets are always closed}
\uthm Let $(X, \dist)$ be a metric space. Let $\mathcal{I}$.
Assume that for every $\alpha \in \mathcal{I}$, we have $C_\alpha \subset X$.
Suppose moreover that for all $\alpha \in \mathcal{I}$, $C_\alpha$ is closed.
Then also
\[
    \bigcap_{\alpha \in \mathcal{I}}\ C_\alpha
\]
is closed.

\subsubsection*{Finite union of closed sets are open}
\uprop Let $(X, \dist)$ be a metric space. Let $C_1, \dots, C_N$ be closed
subsets of $X$. Then the intersection
\[
    C_1 \cap \dots \cap C_N.
\]
is also closed.

\subsubsection*{Cartesian product of open sets}
\uprop Let $C_1, \dots, C_d$ be closed subsets of $\R$. Then
\[
    C_1 \times \dots \times C_d
\]
is a closed subset of $(\R^d, \norm_2)$.

\subsection*{Topological boundary}
\udef Let $(X, \dist)$ be a metric space. Let $A \subset X$.
The \emph{topological boundary} of $A$ is denoted by $\partial A$ and defined as
\[
    \partial A := X \setminus (\setint(A) \cup \setint (X \setminus A))
\]

\subsection{Cauchy sequence}
\udef Let $(X, \dist)$ be a metric space. Let $a : \N \to X$. Then $a$ is a
\emph{Cauchy Sequence} if
\[
    \forall \varepsilon > 0\ \exists N \in \N\ \forall m,n \geq N \left[ \dist(a_m, a_n) < \varepsilon \right].
\]

\subsubsection*{}
\uprop Every Cauchy sequence is bounded.

\subsubsection*{}
\uprop Let $(X, \dist)$ be a metric space. If $a : \N \to X$ converges then
it is also a Cauchy sequence.

\subsubsection*{}
\uprop Let $(X, \dist)$ be a metric space.
Let $a : \N \to X$ be a Cauchy sequence. Assume there exists a subsequence $b$
of $a$ such that $b$ converges to some $q \in X$. Then also $a$ converges to $q$.

\subsection{Completeness}
\udef Let $(X, \dist)$ be a metric space. Let $A \subset X$. We say A is
\emph{complete} if, for evey Cauchy sequence in $A$ is convergent,
with limit in $A$.

\par
We also say the metric space $(X, \dist)$ itself is complete if $X$ is a complete
subset of $X$ in $(X, \dist)$.

\par
Every complete set is also closed.

\subsubsection*{}
\uprop Let $(X, \dist)$ be a metric space. Let $C \subset X$ be complete.
Let $A \subset C$. Then, $A$ is complete if, and only if, $A$ is closed.

\subsubsection*{}
\uprop The metric space $(\R^d, \dist_{\norm_2})$ is complete.

\subsubsection*{}
\uprop Every finite-dimensional normed vector space is complete.


\end{document}